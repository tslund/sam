\documentclass[12pt]{article}
\setlength{\oddsidemargin}{0.0in}
\setlength{\textwidth}{6.0in}
\setlength{\headsep}{-0.75in}
\setlength{\textheight}{9.5in}

\input{macros.tex}

\begin{document}

\section*{Gravity wave solution expressed in phase-aligned coordinates}

The governing equations under the Boussinesq approximation for an inviscid
fluid are
%
\begin{eqnarray}
& & \parderiv{u_j}{x_j} =0\\
& & \parderiv{u_i}{t} + u_j\parderiv{u_i}{x_j} + \parderiv{P}{x_i} =
\frac{\rho}{\rho_0} f_i\\
& & \parderiv{\theta}{t} + u_j\parderiv{\theta}{x_j} = 0
\label{governing}
\end{eqnarray}
%
where $f_i$ is a body force per unit mass and where $P=p/\rho_0$ is the
kinematic pressure.  Now consider the decomposition of
the solution into a background plus a deviation, viz
%
\begin{eqnarray}
u_i & = & \bar{u}_i + u_i^\prime\\
\theta & = & \bar{\theta} + \theta^\prime\\
\rho & = & \bar{\rho} + \rho^\prime\\
P & = & \bar{P} + P^\prime
\label{perturbations}
\end{eqnarray}
%
The undisturbed background is assumed to be in hydrostatic equilibrium so
that
%
\begin{equation}
\parderiv{\bar{P}}{x_i} = \frac{\bar{\rho}}{\rho_0} f_i
\label{hydrostatic}
\end{equation}
%
If this condition is subtracted from the momentum equation, the 
decompositions given above, as well as the thermal expansion law
$\rho^\prime/\rho_0=-\theta^\prime/\theta_0$ are used, the governing 
equations become
%
\begin{eqnarray}
& & \parderiv{u^\prime_j}{x_j} =0\\
& & \parderiv{u^\prime_i}{t} + \bar{u}_j\parderiv{u^\prime_i}{x_j} + 
u^\prime_j\parderiv{u^\prime_i}{x_j} +
\parderiv{P^\prime}{x_i} = -\frac{\theta^\prime}{\theta_0} f_i\\
& & \parderiv{\theta^\prime}{t} + \bar{u}_j\parderiv{\bar{\theta}}{x_j} +
u^\prime_j\parderiv{\bar{\theta}}{x_j} + 
\bar{u}_j\parderiv{\theta^\prime}{x_j} + 
u^\prime_j\parderiv{\theta^\prime}{x_j} = 0
\end{eqnarray}
%
where it has been assumed that the background solution is steady and that
the background velocity is constant.

Now consider a two-dimensional situation ($x$,$z$) where the perturbations
do not depend on the $x$ coordinate and where the normal velocity component
is zero
%
\begin{equation}
u^\prime = u^\prime(z), \quad\quad w^\prime = 0, \quad\quad 
\theta^\prime = \theta^\prime(z), \quad\quad P^\prime = P^\prime(z)
\label{solution:form}
\end{equation}
%
The continuity equation is satisfied identically under these conditions and
all non-linear terms involving products of perturbations vanish.  If it 
is assumed further that background potential temperature gradient is 
perpindicular to the mean wind, then the first advective term in the $\theta$
equation will also vanish.  In component form, the governing equations
thus simplify to
%
\begin{eqnarray}
& & \parderiv{u^\prime}{t} + \bar{w}\parderiv{u^\prime}{z} =
-\frac{\theta^\prime}{\theta_0} f_x\\
& & \parderiv{P^\prime}{z} = -\frac{\theta^\prime}{\theta_0} f_z\\
& & \parderiv{\theta^\prime}{t} + 
u^\prime\parderiv{\bar{\theta}}{x} +
\bar{w}\parderiv{\theta^\prime}{z} = 0
\label{simplified}
\end{eqnarray}
%
Now consider a specific case where the background wind, background
potential temperature gradient, body force, and solution for $\theta^\prime$
are given as
%
\begin{eqnarray}
\bar{u} = U_0\cos\beta \quad & \quad \bar{w} = U_0\sin\beta\\
f_x = g\sin\beta \quad & \quad f_z = -g\cos\beta\\
\parderiv{\bar{\theta}}{x} = -\Gamma\sin\beta \quad & \quad
\theta^\prime = A\frac{\Gamma}{m}\cos(\tilde{m}z-\omega_i t)
\end{eqnarray}
%
where 
%
\begin{equation}
\cos\beta = \frac{m}{\sqrt{k^2+m^2}} \qquad 
\sin\beta = \frac{k}{\sqrt{k^2+m^2}}
\end{equation}
%
and where $k$ and $m$ are the GW solution wavenumbers in ground-based 
coordinates.

Consider first the special case where $U_0=0$.  Using this condition
in addition to those given above, the $\theta^\prime$ equation becomes
%
\begin{equation}
\omega_i A\frac{\Gamma}{m}\sin(\tilde{m}z-\omega_i t) - 
u^\prime\Gamma\sin\beta = 0
\label{thet}
\end{equation}
%
Solving for $u^\prime$
%
\begin{equation}
u^\prime = A\left(\frac{\omega_i}{m}\right)\frac{1}{\sin\beta}
           \sin(\tilde{m}z-\omega_i t) =
A c_i \left(\frac{1}{\cos\beta}\right)\sin(\tilde{m}z-\omega_i t)
\label{uprime}
\end{equation}
%
where $c_i = \omega_i/k$.

The $u^\prime$ equation with $\bar{w}=0$ then evaluates to
%
\begin{equation}
-A\left(\frac{\omega_i^2}{m}\right)\frac{1}{\sin\beta}
\cos(\tilde{m}z-\omega_i t) =  -A\left(\frac{\Gamma g}{\theta_0}\right)
\frac{\sin\beta}{m}\cos(\tilde{m}z-\omega_i t)
\end{equation}
%
The term $(\Gamma g/\theta_0)$ is equal to $N^2$.  The dispersion relation
gives $\omega_i^2 = N^2 k^2/(k^2 + m^2) = N^2\sin^2\beta$.  These results
simplify the above equation to
%
\begin{equation}
-A\left(\frac{N^2}{m}\right)\sin\beta
\cos(\tilde{m}z-\omega_i t) =  -A\left(\frac{N^2}{m}\right)
\sin\beta\cos(\tilde{m}z-\omega_i t)
\end{equation}
%
which validates the dispersion relation used above.

The pressure is found from the $z$ momentum equation
%
\begin{equation}
\parderiv{P^\prime}{z} = A\left(\frac{\Gamma g}{\theta_0}\right)
\left(\frac{\cos\beta}{m}\right)\cos(\tilde{m}z-\omega_i t) =
A c_i^2\left(\frac{1}{\cos^2\beta}\right)m\cos\beta
\cos(\tilde{m}z-\omega_i t)
\end{equation}
%
which integrates to
%
\begin{equation}
P^\prime = A c_i^2\left(\frac{1}{\cos^2\beta}\right)
\left(\frac{m\cos\beta}{\tilde{m}}\right)
\sin(\tilde{m}z-\omega_i t)
\end{equation}
%
The wavenumber $\tilde{m}$ is
%
\begin{equation}
\tilde{m} = \frac{2\pi}{\lambda} = \frac{2\pi}{\lambda_z\cos\beta} =
\frac{m}{\cos\beta}
\end{equation}
%
which simplifies the solution for the pressure to
%
\begin{equation}
P^\prime = A c_i^2 \sin(\tilde{m}z-\omega_i t)
\end{equation}
%
When $U_0\neq 0$, the $x$ momentum equation gives
%
\begin{eqnarray*}
\parderiv{u}{t} & = & -\bar{w}\parderiv{u^\prime}{z} - 
                      \frac{\theta^\prime}{\theta_0}f_x\\
& = & -U_0\sin\beta A c_i \tilde{m} \left(\frac{1}{\cos\beta}\right)
\cos(\tilde{m}z-\omega_i t) - A\left(\frac{\Gamma g}{\theta_0}\right)
\left(\frac{\sin\beta}{m}\right)\cos(\tilde{m}z-\omega_i t)\\
& = & -A c_i\left(\frac{\sin\beta}{\cos\beta}\right)\left( U_0\tilde{m} +
c_i\sqrt{k^2+m^2} \right)\cos(\tilde{m}z-\omega_i t)\\
& = & -A c_i \left(\frac{\sin\beta}{\cos\beta}\right)\left(
\frac{U_0 m}{\cos\beta} + \frac{c_i m}{\cos\beta}\right)
\cos(\tilde{m}z-\omega_i t)\\
& = & -A c_i m\left(\frac{\sin\beta}{\cos^2\beta}\right)
\left( U_0 + c_i\right) \cos(\tilde{m}z-\omega_i t)
\end{eqnarray*}
%
Similarly, the $\theta^\prime$ equation becomes
%
\begin{eqnarray*}
\parderiv{\theta^\prime}{t} & = & -u^\prime\parderiv{\bar{\theta}}{x}
-\bar{w}\parderiv{\theta^\prime}{z}\\
& = & A c_i \left(\frac{1}{\cos\beta}\right)\Gamma\sin\beta\sin(\tilde{m}z-\omega_i t) +
U_0\sin\beta A \Gamma \left(\frac{\tilde{m}}{m}\right)
\sin(\tilde{m}z-\omega_i t)\\
& = & A \Gamma \left(\frac{\sin\beta}{\cos\beta}\right)
\left(c_i + U_0\right) \sin(\tilde{m}z-\omega_i t)
\end{eqnarray*}
%
The solutions for both $u^\prime$ and $\theta^\prime$ become steady if 
%
\begin{equation}
U_0 = -c_i
\end{equation}
%


\end{document}
